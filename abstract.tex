\documentclass[UKenglish]{scrartcl}

\usepackage{babel}

\title{Predicativism and Truth}
\author{Graham E.~Leigh\\\normalsize University of Gothenburg \and Michael Rathjen\\\normalsize University of Leeds}

\begin{document}
\maketitle
\thispagestyle{empty}

% opening
Formal explications of mathematical truth share many similarities with predicativist approaches to foundations of mathematics. 
Like predicativism, many philosophers and mathematicians have put the blame of the liar paradox on the vicious circle of self-reference, and thereby sought `grounded' theories of truth, of which there are two styles: a Tarskian stratification of language into a hierarchy of locally typed truth predicates, or a single `global' truth predicate consistent with an interpretation of truth as a limit of locally justified refinements or reflections.

In this article, we analyse two systems of truth of the above kind that are clearly motivated by a predicativist conception of truth but which, to date, have not been thoroughly examined for their logical and ontological content.
The first is a collection of stratified theories of truth introduced by Nik Weaver~\cite{Wea:pred} that challenge Feferman's thesis of the limits of predicativity.
A novel feature of these theories, unusual in the context of formal truth, is that they assume a base of intuitionistic logic which, Weaver argues, allows predicative justification for his particular hierarchy of truth predicates.
The second notion of truth we examine was proposed by John F.\ Nash in unpublished notes~\cite{Nash} as a framework for overcoming G\"odelean incompleteness via a formal process of `introspection', a form of reflection principle for formal truth.
Via a proof theoretic analysis of these systems we clarify the predicative foundation of these systems relative to the Weyl--Feferman characterisation of predicativity and Weaver and Nash's respective standpoints.

\begin{thebibliography}{2}
	\bibitem{Wea:pred} Nik Weaver, \textit{Predicativity beyond \( \Gamma_0 \)}, \texttt{arXiv:math/0509244}, 2009.
	\bibitem{Nash} John F.~Nash, \textit{Hierarchical Introspective Logics}, unpublished notes, 2010.
\end{thebibliography}

\end{document}